
\documentclass[a4j,oneside,openany]{jsbook}
\setlength{\textwidth}{\fullwidth}
\setlength{\evensidemargin}{\oddsidemargin}
\title{ハンズオン 新生活}
\author{著: 新生活に関する知見がめっちゃある人}
\date{}
\begin{document}
\maketitle

\tableofcontents
\clearpage
\section{はじめに}

\part{新生活の準備}
\chapter{引っ越し準備}
\section{荷造りのコツ}
\section{不用品の処分方法}

\chapter{住居探し}
\section{物件の選び方}
\section{契約書の読み方}

\part{新しい生活の始まり}
\chapter{新居での生活}
\section{家具・家電の揃え方}
\section{生活必需品の買い物リスト}

\chapter{食生活のはじめかた}
\section{料理の基本}
新生活が始まるとはじめは楽しくて料理を行いますが、段々と用意、計画、片付けが面倒くさくなりやらなくなります。よって調味料は絶対に使い切れるものだけに(塩など)しておきましょう。あと、調理器具もフライパン(大)だけあれば大抵なんとかなります。レンジは偉大。
\subsection{調理器具の準備}
\subsection{基本の調理法}

\section{買い物のコツ}
\subsection{食材の選び方}
\subsection{買い物リストの作り方}

\part{具体的な料理}
\chapter{簡単にできる料理}
\section{オムレツ}
\subsection{材料}
\subsection{作り方}
\section{鮭のホイル焼き}
\subsection{材料}
\subsection{作り方}

\chapter{少し手の込んだ料理}
\section{親子丼}
\subsection{材料}
\subsection{作り方}
\section{カルボナーラ}
\subsection{材料}
\subsection{作り方}


\end{document}